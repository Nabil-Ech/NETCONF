\chapter{Abstract}

This internship report explores the realm of network automation and configuration management using \gls{NETCONF} and \gls{YANG} in a virtual network environment with \gls{GNS3}. The primary focus of the internship was to work with Cisco CSR1000v routers, enabling \gls{NETCONF} and \gls{YANG} support, and establishing secure connections for efficient configuration management.

The report details the process of configuring essential network services such as \gls{DHCP}, \gls{NAT}, and \gls{SSH}, enabling seamless communication and secure access to routers. By leveraging the \textit{ncclient} Python library, the report showcases how to interact programmatically with the routers, effectively retrieving and modifying configurations using \gls{NETCONF}.

A significant aspect of the report centers around sub-network configuration, a critical aspect in managing a large institution's network. By assessing the \gls{IP} address requirements of each sub-network, appropriate addresses ranges are allocated to ensure resource optimization and minimal \gls{IP} address wastage.

Furthermore, the report provides a comprehensive demonstration of modifying and retrieving configurations with a focus on maintaining correctness and accuracy. Input validation mechanisms are discussed to handle unexpected inputs gracefully, enhancing the overall reliability of the sub-network allocation code.

The internship experience has equipped the author with valuable skills and knowledge in network automation technologies, Python programming, and the \gls{GNS3} emulation framework. The report concludes with insights into the application of these acquired skills in real-world network engineering scenarios.

Overall, this internship report serves as a valuable resource for understanding \gls{NETCONF} and \gls{YANG}-based network management and offers practical insights into sub-netting allocation techniques, enabling efficient and automated network configuration.


