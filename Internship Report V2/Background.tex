\chapter{Background}
Since network management became an essence for computer networks, the development of related technology has always been coupled with standardization efforts, which are mainly driven by the OSI-based \gls{CMIP} and TCP/IP-based \gls{SNMP}. On one hand, \gls{OSI-SM} has been the most powerful technology but is complicated and expensive, and relies on OSI protocols that have gone out of fashion. On the other hand, \gls{SNMP} is the solution that has been used by most of the industry, but fell victim to its own simplicity: its data modeling capabilities are rudimentary and it does not support configuration management well due to its lack of transaction capabilities.

With the development of computer networks in multiple dimensions (number of devices, time scale for configuration, etc.), configuring large networks becomes an increasingly difficult task. A set of configuration management requirements for IP-based networks are then identified, focusing on network–wide configurations, which provide a level of abstraction above device-local configuration. In this case, the function of configuration data translator must be seriously considered. Other requirements consist of distinguishing between configuration and operational state, providing primitives to support concurrency in a transaction-oriented way, the persistence of configuration changes, security considerations, and so on.

XML-based configuration management is now under hot research, especially using \gls{NETCONF}. Main characteristics of the \gls{NETCONF} protocol are briefly introduced as follows, with a detailed specification in \cite{rfc6241}.

\gls{NETCONF} defines a simple mechanism, through which a network device can be managed, configuration data information can be retrieved, and new configuration data can be uploaded and manipulated. The paradigm that it uses is named Remote Procedure Call (RPC). A key aspect of \gls{NETCONF} is that it allows the functionality of the management protocol to closely mirror the native functionality of the device. Besides, applications can access both the syntactic content and the semantic content of the device's native user interface. In addition, \gls{NETCONF} allows a client to discover the set of protocol extensions supported by a server. These so-called "capabilities" permit the client to adjust its behavior to take advantage of the features exposed by the device.

\gls{NETCONF} Data Modeling Language (\gls{netmod}) Working Group (WG) proposed by the \gls{IETF} aims at supporting the ongoing development of IETF and vendor-defined data models for \gls{NETCONF}, since \gls{NETCONF} needs a standard content layer and its specifications do not include a modeling language or accompanying rules that can be used to model the management information to be configured using \gls{NETCONF}. The main purpose of the \gls{netmod} WG is to provide a unified data modeling language to standardize the \gls{NETCONF} content, by defining a "human-friendly" language and emphasizing readability and ease of use. The defined language is able to serve as the normative description of \gls{NETCONF} data models. Thus, from this point of view, this WG plans to use \gls{YANG} as its starting point for this language \cite{netconf1}.

In the context of this internship project, the \gls{GNS3} network simulation tool is utilized to deploy simple network topologies. \gls{GNS3} allows for the creation of network environments that closely resemble real-world networks. By utilizing \gls{GNS3}, interns can gain practical experience in network deployment and configuration, providing a suitable environment for testing and evaluating the operation of \gls{NETCONF} and its interaction with network devices.

By combining the practical use of \gls{GNS3} and the implementation of \gls{NETCONF}, this internship project aims to explore the deployment and management of network devices using software-defined networking principles.
