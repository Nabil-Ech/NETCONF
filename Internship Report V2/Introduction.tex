\chapter{Introduction}
The heterogeneity of communication networks, along with the ever-changing requirements of services, presents a significant challenge in developing effective techniques for network management. In response to this challenge, \gls{SDN} initiatives have emerged with the aim of providing common and vendor-agnostic control planes for network devices and traffic management. This internship project focuses on understanding the operation of the \gls{NETCONF} \cite{rfc6241}, one of the main protocols for device management, and leveraging the \gls{GNS3} network simulation tool to deploy simple but realistic network topologies. Additionally, a software tool will be developed, preferably in Python, to remotely modify the operation of devices using \gls{NETCONF}.

The management of modern communication networks is essential to ensure optimal performance, scalability, and flexibility. However, the diverse range of network technologies and equipment from different vendors hinders the management process. Traditional network management approaches often rely on proprietary interfaces and protocols, leading to fragmented control planes and limited interoperability.

\gls{SDN} initiatives address these challenges by decoupling the control plane from the underlying network infrastructure. By providing a centralized and programmable control plane, \gls{SDN} allows administrators to define network behavior and policies through software-based controllers. This approach promotes flexibility, agility, and scalability in managing heterogeneous network environments.

One of the key protocols used in \gls{SDN} is \gls{NETCONF}. \gls{NETCONF}, defined by the \gls{IETF}, is a standardized protocol that enables secure and efficient configuration and management of network devices. It provides a programmatic interface for accessing and modifying device configurations, monitoring device state, and retrieving operational data.

During this internship at the University of Cantabria under the supervision of Professor Luis Francisco Diez, the objective was to gain hands-on experience in network management using \gls{SDN} principles and the \gls{NETCONF} protocol. The internship involved leveraging the \gls{GNS3} emulation tool to deploy simple network topologies, simulating real-world network environments. Additionally, a software tool was developed using Python to remotely modify the operation of devices through \gls{NETCONF}.

In the subsequent sections of this report, we will delve into the background of software-defined networking, discuss the objectives and scope of the project, outline the methodology employed, present the results and findings, and conclude with reflections and recommendations. Throughout the report, we will explore the practical implementation of \gls{NETCONF} and its effectiveness in managing network devices in a simulated environment.











