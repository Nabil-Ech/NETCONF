\chapter*{Appendix}
\addcontentsline{toc}{chapter}{Appendix} % Add the appendix to the table of contents

% Include the appendix content
\section{Appendix A: Supporting Documents for the Project Code}


\begin{lstlisting}[style=pythonStyle, caption={Automated Subnet Allocation.}, backgroundcolor=\color{codebackground}]
import ipaddress

def generate_subnets(network, num_routers):
    num_ips_list=[]
    devices = {}
    for i in range(1, num_routers + 1):
        key = int(input(f"Enter the num_ips_router for Device{i}: "))
        num_ips_list.append(key)
        value = f"Device{i}"
        devices[key] = value
    sorted_ips_list = sorted(num_ips_list, reverse=True)
    subnets = []
    mask = []
    mask_dec = []
    power_list=[]
    default_router = []
    power = 0
    index = -1
    left_add=pow(2, 32-ipaddress.IPv4Network("192.168.0.0/" 
    + str(network.netmask)).prefixlen)
    for num_ips in sorted_ips_list:
        for x in range(32-ipaddress.IPv4Network("192.168.0.0/" 
        + str(network.netmask)).prefixlen-1):
            if (pow(2, x)<= num_ips and num_ips < pow(2, x+1)):
                power = x+1
        power_list.append(power)
    for i in range (num_routers):
        if left_add >= pow(2, power_list[i]):
            if i > 0 and power_list[i]==power_list[i-1] :
                subnet_prefix = network.prefixlen + (32-ipaddress.
                IPv4Network("192.168.0.0/" + str(network.netmask))
                .prefixlen-power_list[i])
                subnet = ipaddress.IPv4Network((network.network_address , 
                subnet_prefix))
                new_subnet = subnets[-1]+pow(2, power_list[i]) 
                subnets.append(new_subnet)
                mask.append(subnet.netmask)
                mask_dec.append(subnet_prefix)
                default_router.append(new_subnet + 1)
                left_add = left_add - pow(2, power_list[i])
                print(devices[sorted_ips_list[i]])
                print("Interface IP:", default_router[index])
                print("Subnet Mask:", mask[index], "(",mask_dec[index],")")
                print("DHCP Server Network:", subnets[index])
                print("DHCP Server Default Router:", default_router[index])
                print()
            
            else: 
                subnet_prefix = network.prefixlen + (32-ipaddress.
                IPv4Network("192.168.0.0/" + str(network.netmask))
                .prefixlen-power_list[i])
                subnet = ipaddress.IPv4Network((network.network_address , 
                subnet_prefix))
                new_subnet = subnet.network_address 
                subnets.append(new_subnet)
                mask.append(subnet.netmask)
                mask_dec.append(subnet_prefix)
                default_router.append(new_subnet + 1)
                left_add = left_add - pow(2, power_list[i])
                print(devices[sorted_ips_list[i]])
                print("Interface IP:", default_router[index])
                print("Subnet Mask:", mask[index], "(",mask_dec[index],")")
                print("DHCP Server Network:", subnets[index])
                print("DHCP Server Default Router:", default_router[index])
                print()
        else:
            print("No ip left for", devices[sorted_ips_list[i]])



network_input = input("Enter the network and subnet 
in CIDR notation (e.g., 10.10.0.0/23): ")
num_routers = int(input("Enter the number of routers: "))
network = ipaddress.IPv4Network(network_input)
generate_subnets(network, num_routers)

          
\end{lstlisting}

\begin{lstlisting}[style=pythonStyle, caption={Get-config.}, backgroundcolor=\color{codebackground}]
from ncclient import manager
import xmltodict
import xml.dom.minidom

# Create an XML filter for targeted NETCONF queries
netconf_filter = """
<filter>
  <interfaces xmlns="urn:ietf:params:xml:ns:yang:ietf-interfaces">
    <interface></interface>
  </interfaces>
</filter>"""

print("Opening NETCONF Connection to {}"
      .format("host: ATN-IPS-R-192.168.100.146"))

# Open a connection to the network device using ncclient
with manager.connect(
        host="192.168.1.1",
        port=830,
        username="admin", 
        password="nabil",
        hostkey_verify=False
        ) as m:

    print("Sending a <get-config> operation to the device.\n")
    # Make a NETCONF <get-config> query using the filter
    netconf_reply = m.get_config(source = 'running', filter = netconf_filter)          
\end{lstlisting}
\begin{lstlisting}[style=pythonStyle, caption={Get-capabilities.}, backgroundcolor=\color{codebackground}]
from ncclient import manager
import xmltodict
import xml.dom.minidom

with manager.connect(
        host="192.168.1.10",
        port=830,
        username="admin",
        password="nabil",
        hostkey_verify=False
        ) as m:


        
    for c in m.server_capabilities:
        print (c)         
\end{lstlisting}


\section{Appendix B: Supporting Documents for Router Configuration}

\begin{lstlisting}[style=cliStyle, caption={Configure DHCP.},  backgroundcolor=\color{codebackground}]
configure terminal
ip dhcp pool MyPool // define the pool, the name 
network 192.168.1.0 255.255.255.0 // add the subnet 192.168.2.0/24
default-router 192.168.2.1 // configure the default gateway for this subnet 
exit 
ip dhcp excluded-address 192.168.2.10 192.168.2.20 //to exclud some addresses 
from 10 to 20


// you might as well add some a DNS server and the domain 
dns-server "address" "address"
domain "your domain"
\end{lstlisting}

\begin{lstlisting}[style=cliStyle, caption={Configure NAT.},  backgroundcolor=\color{codebackground}]
conf t
int FastEthernet 0/0
ip nat inside
exit

int FastEthernet 1/0
ip nat outside
exit

ip nat pool <pool-name = POOL> <starting-IP = 192.168.2.1> 
<ending-IP = 192.168.2.100> netmask 255.255.255.0 

access-list <acl-number = 1>
permit <source-ip-network = 192.168.1.0> <wildcard-mask = 0.0.0.255>

//NOTE:The access list configured above matches all hosts from the
 192.168.1.0/24 subnet.*

ip nat inside source
list <acl-number = 1> pool <pool-name = POOL>
exit
end 

ping 192.168.2.5
show ip nat translations
\end{lstlisting}
\begin{lstlisting}[style=cliStyle, caption={Configure NETCONF.},  backgroundcolor=\color{codebackground}]
enable
configure terminal
netconf-yang
netconf-yang feature candidate-datastore
\end{lstlisting}

\begin{lstlisting}[style=cliStyle, caption={Configure SSH.},  backgroundcolor=\color{codebackground}]
enable
configure terminal

//Setting up the device to run SSH & configure SSH server

hostname Router // configure hostname and domain name
ip domain-name Router-domain
crypto key generate rsa modulus 2048 // enter modulus length of at
 least 1024, longer is more secure
ip ssh version 2 // configure the device to run SSHv2

line vty 0 4 // configure the virtual terminal line settings
login local
transport input all // allow all transport connections, include telnet, ssh
end // return to EXEC mode
show ip ssh // show version and conf info SSH server
show ssh // show the status of SSH server connections
configure terminal
username admin password nabil // create user and password for SSH 
client to access
username kimdoanh89 privilege 15 // set privilege privilege level 15 
to the username "admin"

exit
show running-config // verify the entries
copy running-config startup-config // save the entries in the conf fil
// SSH to the router on the local machine (management server)
ssh admin@192.168.1.1 -p 830 -s netconf
password: nabil
\end{lstlisting}