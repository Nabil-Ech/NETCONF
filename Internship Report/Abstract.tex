\chapter{Abstract}

This internship report explores the realm of network automation and configuration management using NETCONF and YANG in a virtual network environment with GNS3. The primary focus of the internship was to work with Cisco CSR1000v routers, enabling NETCONF and YANG support, and establishing secure connections for efficient configuration management.

The report details the process of configuring essential network services such as DHCP, NAT, and SSH, enabling seamless communication and secure access to the routers. By leveraging Python's ncclient library, the report showcases how to interact programmatically with the routers, effectively retrieving and modifying configurations using NETCONF.

A significant aspect of the report centers around subnet allocation, a critical component in managing a large institution's network. By assessing the IP address requirements of each router, appropriate subnets are allocated to ensure resource optimization and minimal IP address wastage.

Furthermore, the report provides a comprehensive demonstration of modifying and retrieving configurations with a focus on maintaining correctness and accuracy. Input validation mechanisms are discussed to handle unexpected inputs gracefully, enhancing the overall reliability of the subnet allocation code.

The internship experience has equipped the author with valuable skills and knowledge in network automation technologies, Python programming, and GNS3 simulations. The report concludes with insights into the application of these acquired skills in real-world network engineering scenarios.

Overall, this internship report serves as a valuable resource for understanding NETCONF and YANG-based network management and offers practical insights into subnet allocation techniques, enabling efficient and automated network configuration.

